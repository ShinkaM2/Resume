%%%%%%%%%%%%%%%%%%%%%%%%%%%%%%%%%%%%%%%%%
% Medium Length Professional CV
% LaTeX Template
% Version 2.0 (8/5/13)
%
% This template has been downloaded from:
% http://www.LaTeXTemplates.com
%
% Original author:
% Rishi Shah 
%
% Important note:
% This template requires the resume.cls file to be in the same directory as the
% .tex file. The resume.cls file provides the resume style used for structuring the
% document.
%
%%%%%%%%%%%%%%%%%%%%%%%%%%%%%%%%%%%%%%%%%

%----------------------------------------------------------------------------------------
%	PACKAGES AND OTHER DOCUMENT CONFIGURATIONS
%----------------------------------------------------------------------------------------
\documentclass[10pt]{resume} % Use the custom resume.cls style

\usepackage[left=0.75in,top=0.6in,right=0.75in,bottom=0.6in]{geometry} % Document margins
\newcommand{\tab}[1]{\hspace{.2667\textwidth}\rlap{#1}}
\newcommand{\itab}[1]{\hspace{0em}\rlap{#1}}

\name{Shinka Mori} % Your name
% \address{Resume}

\address{ 408 East Springfield Ave. Champaign IL,61820} % Your address
%\address{123 Pleasant Lane \\ City, State 12345} % Your secondary addess (optional)
\address{(919)-808-8177 \\ mori.shinka@gmail.com} % Your phone number and email

\begin{document}

%----------------------------------------------------------------------------------------
%	EDUCATION SECTION
%----------------------------------------------------------------------------------------

\begin{rSection}{Education & Certifications}

\\{\bf  University of Illinois at Urbana-Champaign} \hfill {\em May 2021 3.58/4.00} 
\\ Bachelor of Science in Computer Science and Linguistics    
\\ Minor in Mathematical Statistics
\\{\bf Coursera}\hfill {\em February 2020} 
\\ DeepLearning.AI TensorFlow Developer by deeplearning.ai 

\\{\bf Relevant Coursework}
\\ Computational Linguistics, Deep Learning, Machine Learning, Machine Translation, Computational Morphology, Bayesian Statistics, Applied Statistics
\end{rSection}


%----------------------------------------------------------------------------------------
%	WORK EXPERIENCE SECTION
%----------------------------------------------------------------------------------------

\begin{rSection}{Work Experience}
\begin{rSubsection}{United Healthcare/Optum}{June 2020 - August 2020}{Technology Development Intern}{Raleigh, NC}
\item Worked in an Agile team of eight people to develop a dashboard to track and predict COVID-19 cases within United Healthcare offices
\item Used Jenkins, Docker, Openshift and MongoDB to set up a CI/CD pipeline for backend
\item Developed and maintained the Machine Learning pipeline using Python, Jenkins, OpenShift, Docker and MongoDB, which fetched data daily, preprocessed them, stored them in MongoDB, and created a weekly forecast using multiple Machine Learning models
\item Contributed to the web application by using Angular to create visuals for COVID-19 case growth
\end{rSubsection}

\begin{rSubsection}{Intelligent Medical Objects}{May 2019 - July 2019}{Medical Terminology Analyst}{Champaign, IL}
\item \small Analyzed and processed over a thousand clinical terminology entries into IMO’s medical dictionary that is used in healthcare facilities worldwide
\item Reviewed, organized, and documented SQL queries used for data analysis by the Knowledge Operations team


\end{rSubsection}
\begin{rSubsection}{PROMT}{July 2018}{Machine Translation Intern}{Saint Petersburg, Russia}
\item \small Worked in a four person team to ensure quality assurance for the translations produced by the company’s Machine Translation program
\item Improved the Russian - English dictionaries for their Machine Translation Program

\end{rSubsection}

\end{rSection}


%	ACTIVITIES AND LEADERSHIP
%----------------------------------------------------------------------------------------

\begin{rSection}{Activities and Leadership}
\begin{rSubsection}{CS 233: Computer Architecture}{January 2021 - Present}{Course Assistant}{Champaign, IL}
\item \small Assist students in CS 233 by helping them through lecture handouts during lecture
\item \small Holding office hours for two hours per week to help students with their weekly coding assignments in C, MIPS and Verilog
\end{rSubsection}

\begin{rSubsection}{CS 421: Programming Languages and Compilers}{January 2021 - Present}{Course Assistant}{Champaign, IL}
\item \small Assist students in CS 421 by attending lecture and helping students understand lecture activity
\item \small Holding office hours for an hour per week to help students with their weekly coding assignments and understand lecture materials
\end{rSubsection}

\begin{rSubsection}{Research Assistant}{January 2021 - Present}{Machine Traslation}{Champaign, IL}
\item \small Creating a Japanese-English bidirectional neural machine translator with a Ph.D student to be submitted for EMNLP2020's News Shared task under supervision of Dr. Lane Schwartz

\end{rSubsection}

\begin{rSubsection}{Google Developer Student Club}{October 2020 - Present}{Club Member}{Champaign, IL}
\item \small Participate in biweekly meetings to develop applications using Google products
\item \small Creating a dashboard to list nearby COVID-19 testing locations and its wait times
\end{rSubsection}


\begin{rSubsection}{Research Assistant}{September 2019 - December 2020}{Information Sciences}{Champaign, IL}
\item \small Programming a Java version of SemRep, a toolkit used to extract semantic information from biomedical texts using Natural Language Processing. Modeled off of SemRep, produced by the National Institute of Health 
\item Used Python to preprocess the data used for a submission for the CL-Scisumm shared task, which uses NLP to analyze citation in research papers
\end{rSubsection}

\begin{rSubsection}{CS 125: Introduction to Computer Science}{January 2018 - December 2019}{Course Assistant}{Champaign, IL}
\item \small Advised over thirty students per week enrolled in CS 125 on their weekly programming assignments in Java
\item Assisted CS 199, a supplementary course for CS 125, to explain the topics covered in class in detail
\item Assisted office hours for four hours per week to answer questions on Java and coding concepts
\end{rSubsection}

\begin{rSubsection}{Research Assistant}{September 2019 - Present}{Phonology}{Champaign, IL}
\item \small Assisted data processing for a Ph.D student on their research on Japanese Vowel Devoicing
\item Produced code to edit scripts used to process sound files.
\end{rSubsection}
\begin{rSubsection}{League of Linguists}{September 2019 - Present}{President, Member}{Champaign, IL}
\item \small Participated at the monthly Linguistics Faculty meeting as a representative of the undergraduate students in the Linguistics major
\item Participated in Departmental Committees (Capricious Grading Committee, Grievances Committee)
\item Led a subgroup to teach students Python and solve Computational Linguistics problems
\end{rSubsection}
% \begin{rSubsection}{UIUC Electrophysiology and Language Processing Lab}{January 2019 - May 2019}{Lab Assistant}{Champaign, IL}
% \item \small Worked under a psycho-linguistics professor for a project on L2 learners and pronoun resolution
% \item Produced and organized stimuli for a psycholonguistics project
% \end{rSubsection}







\end{rSection}

%----------------------------------------------------------------------------------------
%	TECHNICAL STRENGTHS SECTION
%----------------------------------------------------------------------------------------

\begin{rSection}{Technical Strengths}

\begin{tabular}{ @{} >{\bfseries}l @{\hspace{6ex}} l }
Programming Languages \ & Java, Python, C++, C, R \\
Tools \ & Latex, AWS, Google Cloud, TensorFlow, Pytorch\\
\end{tabular}

\end{rSection}


\end{document}
